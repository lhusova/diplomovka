\documentclass[thesismargins, thesislinespacing]{rnthesis}
\usepackage[slovak]{babel}
\usepackage[utf8]{inputenc}
\usepackage[]{graphicx}
\usepackage{nomencl}
\usepackage{lineno} 
\usepackage{caption}
\usepackage{subcaption}

\title{}
\author{Lucia Anna Husová}
\typprace{Diplomová}
\rok{2018}
\miesto{Košice}
\odbor{Fyzika}
\podakovanie{
Na tomto mieste sa chcem poďakovať vedúcemu mojej práce doc. RNDr. Marekovi Bombarovi, PhD. za odborné rady a   osobný prístup.  Taktiež sa chcem poďakovať priateľovi Martinovi a rodičom za osobnú podporu.  	
  
} 
\veduci{RNDr. Marek Bombara, PhD.}
\pracovisko{Ústav fyzikálnych vied, Katedra jadrovej a subjadrovej fyziky}

\abstract{
	
	The low energy jets are hardly identified in heavy ions collisions due to high background and the possible shape deformation. Therefore the two-particle correlation method is used to study these low energy jets in heavy ions collisions.\\
	In this thesis, the results of two-particle corellation method applied on the proton-proton collisions data at 13~TeV collected by the ALICE experiment at the LHC are presented. The yields for near-side and away-side peak for identified and unidentified trigger particles as a function of transversal momentum of trigger particles were measured. The hadrons $K^0_S$, $\Lambda$ a $\bar{\Lambda}$ were chosen as identified trigger particles, because of their easy indentification at high transversal momentum, thanks to their decay topology.  

}


\abstrakt{
	
	Nízkoenergetické jety vzniknuté v zrážkach ťažkých iónov sú ťažko identifikovateľné vzhľadom na vysoké pozadie a možnú deformáciu ich tvaru, preto sa na ich štúdium najmä v zrážkach ťažkých iónov využíva metóda dvojčasticových korelácií.\\
	V tejto práci sú prezentované výsledky metódy dvojčasticových korelácií aplikovanej na dáta z protónovo-protónových zrážok pri energii 13 TeV zozbieraných na experimente ALICE na urýchľovači LHC. Merané boli výťažky pre priľahlý aj protiľahlý pík pre identifikované aj neidentifikované trigrovacie častice v závislosti od priečnej hybnosti trigrovacích častíc. Za identifikované trigrovacie častice boli zvolené podivné hadróny $K^0_S$, $\Lambda$ a $\bar{\Lambda}$, ktoré sa ľahko identifikujú aj pre vysoké priečne hybnosti vďaka svojej rozpadovej topológii.

\newpage
}

\makeglossary


\bibliographystyle{alpha}
%\linenumbers 
\begin{document}
\begin{center}
	{\Large Univerzita Pavla Jozefa Šafárika v Košiciach} \\
	%\end{center}
	%\begin{center}
	{\Large Prírodovedecká fakulta} 
\end{center}

\vspace*{2cm}

\begin{figure}[htbp!]
	\begin{center}
		\includegraphics[width=3cm]{./Obrazky_praca/logo-pf-upjs-cb.jpg}
	\end{center}
\end{figure}

\vspace*{2cm}

\begin{center}
	%\vspace*{7cm}
	{\LARGE\bf Uhlové korelácie s podivnými časticami}
\end{center}

\begin{center}
	{\large Diplomová práca}
\end{center}

\vspace*{5cm}
\begin{flushleft}
{\bf Študijný odbor:}{ Jadrová a subjadrová fyzika} \\
{\bf Školiace pracovisko: }{Katedra jadrovej a subjadrovej fyziky}\\
{\bf Vedúci práce: }{ RNDr. Marek Bombara, PhD.}\\
\end{flushleft}
 
 \vspace*{2cm}
 \begin{flushleft}
{\large Košice 2018}
\hspace*{5cm}
{\large Bc. Lucia Anna Husová}
\end{flushleft}

\thispagestyle{empty}
\newpage

\maketitle
\newpage
\tableofcontents
\newpage




\chapter*{Úvod}

Ani v súčasnej dobe nevieme vysvetliť množstvo fundamentálnych otázok, preto si fyzika vysokých energií stále udržuje primát jednej z najaktívnejších oblastí fyziky. Na tento účel bol postavený najväčší urýchľovač častíc LHC (Large Hadron Collider) v laboratóriu CERN, ktorý nie je zameraný len na štúdium novej fyziky, ale aj na testovanie správnosti Štandardného modelu elementárnych častíc. Na LHC sa okrem protónovo-protóvých zrážok skúmajú aj zrážky jadier olova, pri ktorých  sa vytvára kvarkovo-gluónová plazma (QGP), o ktorej sa predpokladá, že ňou bol tvorený Vesmír tesne po Veľkom tresku. 

Vlastnosti QGP sa vo väčšine prípadov skúmajú nepriamo prostredníctvom \-had\-ró\-nov, ktoré obsahujú kvarky a gluóny pochádzajúce z plazmy. Jedným z populárnych \-spô\-so\-bov sú tzv. "hard probes". Ide o rýchle partóny, ktoré interagujú s plazmou. Tie následne rozpoznávame v detektoroch ako spŕšky hadrónov (jety). Jety sa dajú skúmať \-pria\-mo alebo nepriamo, napr. metódou dvojčasticových korelácií. Jednou zo zaujímavých otázok v oblasti "hard probes" \-je, ako závisí interakcia rýchleho partónu s plazmou od farebného náboja. Partón môže niesť farbu (kvark), antifarbu (antikvark) alebo kombináciu oboch (gluón). Keďže kvarkové a gluónové jety majú rôzne vlastnosti a v~priemere pozostávajú z rôznych hadrónov, táto problematika sa dá študovať pomocou korelácií s identifikovanými časticami.  

\chapter{Kvantová chromodynamika}

Po objave vnútornej štruktúry protónov v 60-tych rokoch minulého storočia v laboratóriu SLAC (Stanford Linear Accelerator Center) bolo potrebné vytvoriť teoretický model popisujúci správanie sa novoobjavených častíc - partónov, ktoré boli neskôr priradené k teoreticky predpovedaným kvarkom a gluónom. Na pochopenie interakcie medzi partónmi vznikla kvantová chromodynamika (QCD), ktorá bola sformulovaná na základe symetrií, podobne ako staršia kvantová elekrodynamika (QED). V QED je zdrojom vzájomného pôsobenia elektrický náboj. Jeho analógiou v QCD je farebný náboj ako zdroj silnej interakcie, ktorý môže nadobúdať šesť stavov (modrý, červený, zelený, antimodrý, antičervený, antizelený) \footnote{druhy farebného náboja nesúvisia s farbami viditeľnej časti elektromagnetického spektra}.

\section{Farebný náboj}
Farba, ako nové kvantové číslo, bola zavedená kvôli Pauliho vylučovaciemu princípu, ktorý hovorí, že fermióny (častice s polčíselným spinom) v jednom kvantovo-me\-cha\-nic\-kom systéme nemôžu mať všetky kvantové čísla rovnaké. Vedelo sa však o existencii častíc, ktoré sa skladajú z kvarkov s rovnakými dovtedy známymi kvantovými číslami (napr. $\Delta^{++}$ (uuu) alebo $\Omega^{-}$ (sss)). Ponúkalo sa niekoľko vysvetlení:
\begin{itemize}
	\item Neplatí Pauliho vylučovací princíp.
	\item Kvarky nie sú fermióny, ale bozóny s celočíselným spinom.
	\item Existuje ďalšie kvantové číslo.
\end{itemize}
Keďže Pauliho princíp je založný na kauzalite, nemožno o ňom prehlásiť, že je neplatný. Preto sa najpriateľnejšou možnosťou ukázalo zavedenie nového kvantového čísla, farby. Vlnová funkcia kvantovo-mechanického systému sa stáva opäť antisymetrickou a tým pádom zostáva platný Pauliho vylučovací princíp~\cite{1}:
\begin{equation}
	\Psi_{TOTAL}=\Psi_{SPACE}*\Psi_{SPIN}*\Psi_{FLAVOUR}*\Psi_{COLOUR}, 
\end{equation}   
kde $\Psi_{SPACE}$ je priestorová časť vlnovej funkcie, $\Psi_{SPIN}$ je spinová časť, $\Psi_{FLAVOUR}$ je časť popisujúca druh karku a $\Psi_{COLOUR}$ je časť určujúca farebný náboj.

Pri zohľadnení farebného náboja sa vyriešili aj iné dovtedy nepochopené rozdiely medzi výpočtami a výsledkami experimentov, ako napríklad účinný prierez elektromagnetického rozpadu $\pi^0$ mezónu na dva fotóny alebo $\tau$ leptónu na hadróny. 

Farebný náboj však nenesú len samotné kvarky, ale aj nosiče silnej interakcie - gluóny, ktoré sú dvojfarebné (sú nosičom farby a antifarby). Preto môžu interagovať ako s kvarkami, tak aj so sebou navzájom. Jedným z dôsledkov tejto vlastnosti je, že farebné častice nemôžu existovať ako voľné častice, a teda farebný náboj je uväznený vnútri hadrónov.

Samointerakcia gluónov vyplýva aj z vlastností špeciálnej unitárnej grupy SU(3), ktorá popisuje transformácie v rámci QCD. Keďže táto grupa nie je Abelovská, jednotlivé transformácie spolu nekomutujú a teda v Lagrangiáne popisujúcom silnú interakciu vznikajú samointerakčné členy gluónov (interakcia 3 alebo 4 gluónov). Grupa SU(3) má 8 voľných parametrov, preto sa vyžaduje existencia 8 rôznych gluónov. Gluóny môžu vrámci tejto grupy tvoriť oktet alebo singlet. Singlet však nie je pozorovaný. Jeho existencia by viedla ku vzniku ďaleko-dosahovej silnej interakcie a vlastnosťami by sa podobal na fotón. 

Z experimentov sa zistilo, že v prírode existujú iba navonok bezfarebné multiplety kvarkov tvoriace hadróny, a teda všetky z nich sú singlety (nemenia farebný náboj počas interakcie). Sú to:
\begin{itemize}
	\item mezóny - kombinácia farby a antifarby ($q \bar q$)
	\item "biele" baryóny - kombinácia troch rôznych farieb ($qqq$) a antibaryóny - kombinácia troch rôznych antifarieb ($\bar q \bar q \bar q$)
	\item exotické stavy - tetrakvarky ($q \bar q q \bar q$), pozorované v apríli 2014 \cite{tetra} a pentakvarky ($qqqq \bar q $), pozorované v júli 2015 na experimente LHCb \cite{2}
\end{itemize}

\section{Asymptotická sloboda}
Ďalšia vlastnosť QCD, ktorá je dôsledkom samointerakcie medzi gluónmi, je asymptotická sloboda. 
Veľkosť náboja v silnej, ale aj v elekromagnetickej interakcii závisí od vzdialenosti, z ktorej náboj študujeme. Pri štúdiu elektromagnetickej interakcie na veľmi malých vzdialenostiach nemožno zanedbávať kvantovo-mechanické efekty, ako je napr. polarizácia vákua. V okolí elektrónu sa vytvoria virtuálne páry elektrónov a pozitrónov a kvôli ich  orientácii vznikne v okolí pôvodného elektrónu oblak kladného náboja. Z tohto dôvodu pri väčších vzdialenostiach nameriame tzv. "renormalizovaný" \-e\-lektrický náboj, ktorý má menšiu hodnotu ako pôvodný čistý náboj.

Podobný proces nastáva aj pri silnej interakcii. V okolí farebne nabitého kvarku sa vytvárajú páry kvark-antikvark a gluóny. Na rozdiel od elektromagnetickej interakcie, kde fotóny na seba navzájom nepôsobia, gluóny na seba pri silnej interakcii pôsobia. To vedie k tomu, že sila silnej interakcie sa so zväčšujúcou vzdialenosťou zväčšuje. Na malých vzdialenostiach môžeme povedať, že silná interakcia asymptoticky zaniká (asymptotická sloboda), aj keď je stále väčšia ako elektromagnetická, a z kvarkov a gluónov sa z pohľadu silnej interakcie stávajú voľné častice. 

Táto vlastnosť sa tiež prejavuje pri vysokoenergetických nepružných zrážkach. Čím majú zrážajúce sa častice vyššiu hybnosť, tým je silná interakcia pôsobiaca medzi nimi slabšia (obr.~\ref{alfa}).
\begin{figure}
	\centering
	\includegraphics[scale=0.4]{./Obrazky_praca/zrazka.png}
	\caption{Závislosť konštanty silnej interakcie $\alpha_s$ od energie jednotlivých procesov Q \cite{3}}
	\label{alfa}
\end{figure}


\section{Jety}

Kvarky a gluóny sú uväznené vnútri hadrónov\footnote{Uväznenie kvarkov v hadrónoch je experimenálny fakt, ale na rozdiel od asymptotickej slobody zatiaľ nie je vysvetlené v QCD.}, a teda nemôžeme pozorovať vlastnosti kvarkov a gluónov priamo. Pri udelení dostatočnej energie partónu v hadróne sa táto energia premení na spŕšku hadrónov v jednom smere (tzv. jet), v ktorej sa nachádza hadrón obsahujúci pôvodný partón. V experimentoch je jet definovaný ako skupina energe\-tických hadrónov pohybujúcich sa v jednom smere, ktoré je možné ohraničiť mysleným kužeľom so stredom určeným smerom pohybu originálneho partónu (tzv. jet axis) a s polomerom R~\cite{4}:
\begin{equation}
\sqrt{\Delta \phi^2 + \Delta \eta^2}<R=1,
\end{equation}
kde $\Delta \phi$ je rozdiel azimutálnych uhlov hadrónu v jete a pôvodného partónu, $\Delta \eta$ je rozdiel pseudorapidít\footnote{Pseudorapidita $\eta$ je funkciou polárneho uhla $\theta$: $\eta = - \ln( \tan (\frac{\theta}{2}))$.} hadrónu v jete a pôvodného partónu.

\subsection{Lundský strunový model}
Kým interakcie partónov pri vysokých energiách sú dobre popísané pomocou QCD, na popis vytvárania hadrónov v jetoch si musíme pomôcť fenomenologickými \-mo\-del\-mi. Jeden zo základných modelov, ktorý je prítomný vo väčšine Monte Carlo (MC) generátorov vo fyzike vysokých energií, je Lundský strunový model.

Pre potenciál viazaného stavu kvarku a antikvarku platí:
\begin{equation}
	V(r) \approx - \frac{4}{3} \frac{\alpha_s}{r} + \kappa r \approx - \frac{0.13}{r} + r~\cite{5}
\end{equation}

Na veľmi malých vzdialenostiach ($r\rightarrow0$) prevláda potenciál Coulombovského poľa (obr.~\ref{elpole}). Pri vzájomnom vzďaľovaní sa kvarkov začne prevládať lineárna časť potenciálu a na rozdiel od elekromagnetického poľa sa kvôli samointerakcii gluónov budú siločiary k sebe približovať a vytvoria tvar podobný trubici (obr.~\ref{chrompole}).

Práca vykonaná na oddelenie kvarkov sa mení na energiu poľa. Ak je vzdialenosť medzi týmito dvoma kvarkami dostačujúca, z energie poľa sa vytvorí nový pár častíc kvark-antikvark (obr.~\ref{jet}). Takto to pokračuje ďalej, pokiaľ majú vzďaľujúce sa partóny dostatok energie na vytvorenie nového páru kvark-antikvark. Celý proces pripomína trhanie struny, odkiaľ pochádza aj názov modelu.

\begin{figure}[hbtp!]
	\centering
	\includegraphics[scale=0.08]{./Obrazky_praca/el_pole.png}
	\caption{Siločiary elektrického poľa sa od seba so zväčšujúcou sa vzdialenosťou vzďaľujú~\cite{6}.}
	\label{elpole}
\end{figure}
\begin{figure}[hbtp!]
	\centering
	\includegraphics[scale=0.25]{./Obrazky_praca/chromo_pole.png}
	\caption {Siločiary chromodynamického poľa sa k sebe so zväčšujúcou sa vzdialenosťou približujú~\cite{7}.}
	\label{chrompole}
\end{figure}

\begin{figure}[hbtp!]
	\begin{center}
	\includegraphics[scale=0.5]{./Obrazky_praca/jet.png}
	\caption{Tvorba jetu pripomína trhanie struny. Čím viac od seba naťahujeme konce struny, tým väčšiu silu musíme vynaložiť. Pri prasknutí struny získame dve nové struny~\cite{8}.}
	\label{jet}
	\end{center}
\end{figure}  

\section{Kvarkovo-gluónová plazma}

Pri extrémnych podmienkach (vysoká teplota, vysoká hustota) sa môžu kvarky a~gluó\-ny zbaviť svojho uväznenia v hadrónoch. Keďže tieto pat
róny sa môžu pohybovať voľne v rámci určitého objemu, nazýva sa takýto stav hmoty aj kvarkovo-gluónová plazma (QGP). Predpokladá sa, že tesne po Veľkom tresku bola teplota dostatočne vysoká, aby bol celý Vesmír tvorený QGP. V súčasnosti by sa QGP mohla nachádzať v jadrách niektorých neutrónových hviezd, kde by mala byť dostatočná hustota a teplota na to, aby sa kvarky a gluóny uvoľnili z neutrónov.

\begin{figure}[hbtp!]
	\begin{center}
		\includegraphics[scale=0.5]{./Obrazky_praca/phasDiag.png}
		\caption{ Fázový diagram QGP \cite{phasDiagram}}
		\label{fazDiag}
	\end{center}
\end{figure}  

QGP vzniká pri vysokých teplotách alebo pri vysokom baryónovom potenciáli ako je vidieť na fázovom diagrame na obr.~\ref{fazDiag}. QGP je preto možné vytvoriť aj laboratórnych podmienkach pri zrážkach ťažkých iónov na urýchľovačoch RHIC a LHC. V mieste zrážky, ešte pred vytvorením hydrodynamickej rovnováhy, vzniká hustá hmota nazývaná glasma~\cite{glasma}.  Keďže ióny sa zrážajú s vysokou energiou, v mieste zrážky sa nachádza vysoká hustota energie a vytvorí sa tlakový gradient, vďaka ktorému hmota prudko expanduje a vytvorí sa “little bang”. Za veľmi krátky čas ($\tau_{form}<1$ fm/c)  sa novovytvorené kvarky a gluóny dostanú do termodynamickej rovnováhy a vytvoria kvapku QGP, ktorej vlastnosti sa podobajú na vlastnosti ideálnej kvapaliny. Na popis časovo-priestorového vývoja QGP je Lagrangian QCD prílž zložitý, a preto sa používajú hydrodynamické modely založené na lokálnej termodynamickej rovnováhe (stavové rovnice a relativistická hydrodynamika).

Systém bez vonkajšieho tlaku rýchlo expanduje a chladne a po čase 10 fm/c a pri kritickej teplote 160~MeV\footnote{Vo fyzike elementárnych častíc sa na určenie teploty používajú jednotky eV, kde $\frac{1eV}{k}= 11605K$, a teda 160MeV zodpovedá teplote $1.8\times10^{12}K$.}, ktorá je blízko teplote fázového prechodu, sa partóny formujú do hadrónov. Nastáva tzv. chemický "freeze-out\-", t.j. počet jednotlivých druhov hadrónov sa ustáli a vytvorí sa hadrónový plyn, kde hadróny môžu stále interagovať pružne. Úplne interagovať prestávajú pri teplote 120~MeV, kedy nastáva kinematický "freeze-out" \-a hadróny sa rozletia do detektorov (obr.~\ref{QGP}) . Keďže QGP existuje taký krátky čas, jej vlastnosti je možné skúmať iba nepriamo pomocou hadrónov\footnote{Výnimkou sú napr. fotóny vznikajúce priamo v plazme, keďže ich stredná voľná dráha je väčšia ako rozmery QGP vzniknutej pri zrážke dvoch jadier.}.  

\begin{figure}[hbtp!]
	\begin{center}
		\includegraphics[scale=0.4]{./Obrazky_praca/QGP1.png}
		\caption{ Časový a priestorový vývoj QGP \cite{vyvoj}}
		\label{QGP}
	\end{center}
\end{figure}  

Pomocou QGP možno skúmať priamu interakciu častíc s  farebným nábojom, pretože je tvorená voľnými partónmi nesúcimi farebný naboj, ktorý je inak uväznený v hadrónoch, a overovať tak predpovede QCD. 

\subsection{Znaky QGP} 

Vznik QGP pri zkážke ťažkých iónov je indikovaný viacerými znakmi ako je zhášanie jetov, kedy jety pri prechode QGP strácajú energiu vplyvom elastických zrážok alebo radiatívnych procesov (gluon Bremsstrahlung). Ďalším indikátorom je topenie quarkonie\footnote{Mezón tvorený kvarkom a antikvarkom toho istého druhu, napr. $J/ \psi$ tvorený párom $c \bar{c}$ } , kedy je interakcia medzi kvarkom a antikvarkom v mezóne tienená ostatnými kvarkami a gluónmi v QGP a quarkonium sa “roztopí”. 

Jedným z ďalších znakov QGP je aj zvýšená produkcia podivnosti oproti protónovo-protónovým či elektónovo-pozitrónovým zrážkam. Zvýšenie je natoľko veľké, že nemôže byť spôsobené iba nepružnými zrážkami novovytvorených hadrónov, ale jeho pôvod je v samotnej QGP. Táto zvýšená produkcia je spôsobená viacerými faktormi:
\begin{itemize}
	\item \textbf{Relatívne potlačnie produkcie podivnosti v protónovo-protónových zážkach.} Pri týchto zrážkach vytvorenie podivného hadrónu musí byť kompenzované kvôli zachovaniu podivnosti tvorbou hadrónu s opačnou podivnosťou. Tvorba týchto dvoch hadrónov musí nastať v tom istom čase a na tom istom mieste.  Fázový priestor konečného stavu je teda obmedzený a je potrebná väčšia energia.
	\item \textbf{Vysoká baryónova hustota v QGP.} Centrálna časť fireballu (nebude mi vyčítane toto slovo?) je tvorená prevažne ľahkými kvarkami ($u$, $d$) a teda ďalšia produkcia týchto kvarkov je potlačná kvôli  "Pauli blocking". Z tohto dôvodu je relatívne zvýšená produkcia $s$ kvarku.
	\item \textbf{Prítomnosť gluónov v QGP.} Fúzia gluónov $gg\rightarrow s \bar{s}$ tvorí najväčší príspevok k tvorbe podivnosti v QGP, hoci interakcia ľahkých kvarkov $q \bar{q} \rightarrow s \bar{s}$ tvorí nezanedbateľný príspevok.
\end{itemize}

Tvorba podivnosti z termálnych gluónov bola predpovedaná ako jedna zo \-zák\-lad\-ných charakteristík QGP. Pár $s \bar{s}$ vzniká na jednom miete a v jednom čase v QGP, ale podivné hadóny sú pozorované akoby vznikali na opačných koncoch fireballu štatisticky nezávisle na sebe. Fázový priestor nie je obmedzený ako v protónovo-protónových zrážkach. To môže byť vysvetlené iba silnou difúziou v médiu, teda prítomnosťou QGP.

Čím sú hadróny viac podivnejšie (obsahujú väčší podiel podivných kvarkov resp. antikvarkov), tým je zvýšená produkcia viditeľnejšia. To bolo namerané na experimente WA97 na urýchľovači SPS v CERNe (obr.~\ref{zvysenie}).


\begin{figure}[hbtp!]
	\begin{center}
		\includegraphics[scale=0.4]{./Obrazky_praca/zvysenie.png}
		\caption{ Zvýšenie výťažkov podivných hadrónov v jadrovo-jadrových zrážkach ako funkcia ich celkovej podivnosti \cite{zvysenie}}
		\label{zvysenie}
	\end{center}
\end{figure}  



\chapter{Dvojhadrónové korelácie na urýchľovačoch RHIC a LHC}

\section{Metóda dvojčasticových korelácii}
Metóda dvojčasticových (dihadrónových) korelácii je jednou z nepriamych metód na štúdium jetov. Využíva sa najmä v jadrovo-jadrových zrážkach, kde kvôli \-mo\-di\-fi\-ká\-cii jetov vplyvom QGP nemožno použiť priame metódy, ktoré sú používané pri protónovo-protónových zrážkach.

Spočíva v zvolení intervalu priečnej hybnosti\footnote{Priečna hybnosť je definovaná ako: $p_T=\sqrt{p_x^2+p_y^2}$, pričom rovina $xy$ je kolmá na os zväzku v experimente.}, v ktorom sa hľadá častica s vysokou priečnou hybnosťou, tzv. trigrovacia častica ("trigger particle"), o ktorej sa predpokladá, že je to hadrón obsahujúci pôvodný partón. Tiež je potrebné zvoliť interval nižších hybností pre asociované častice ("\-associated particles"). Následne sa robia rozdiely v \-a\-zi\-mu\-tál\-nom uhle $\phi$ a pseudorapidite $\eta$ pre rôzne intervaly priečnej hybnosti trigrovacích a asociovaných častíc:

\begin{equation}
\Delta \phi = \phi_{trig} - \phi_{asoc}
\end{equation}

\begin{equation}
\Delta \eta = \eta_{trig} - \eta_{asoc}
\end{equation}

Tieto rozdiely sa plnia do dvojrozmerných histogramov, z ktorých sa robia projekcie na jednotlivé osi. V projekcii na $\Delta \varphi$ sa vytvoria dva píky zodpovedajúce dvom jetom. Keďže pred zrážkou bola celková priečna hybnosť nulová, potom je vytvorenie jedného jetu sprevádzané vytvorením jetu na protiľahlej strane v rovine $xy$. Jeden pík je teda so stredom v 0, nazýva sa priľahlým píkom (v angličtine "near-side peak"), a je spôsobený pármi, kde asociované častice pochádzajú z toho istého jetu ako trigrovacia častica. Druhý pík sa nachádza v okolí $\pi$ a je spôsobený takými pármi častíc, kde asociovaná častica pochádza z jetu, ktorý je oproti jetu, z ktorého je trigrovacia častica. Preto sa nazýva protiľahlý pík (v angličtine ''away-side peak'')(obr.~\ref{kor}). Iná situácia nastáva v pozdĺžnom smere. Partóny v protónoch vo zväzku môžu mať v princípe rôznu pozdĺžnu hybnosť, a teda ak vidíme v detektore jet v pozdĺžnom smere, druhý nemusí byť na protiľahlej strane. Výberom trigrovacej častice vyberáme jety, ktoré sa nachádzjú v detektore a ich náprotivky s nimi nebudú zvierať uhol $180^\circ$. Preto sa v projekcii na $\Delta \eta$ vytvorí jediný pík so stredom v 0.

\begin{figure}[hbtp!]
	\begin{center}
		\includegraphics[width=\textwidth]{./Obrazky_praca/dijetcorrelations.png}
		\caption{Vľavo: Schéma polohy trigrovacej a asociovanej častice. Vpravo: Poloha priľahlého a protiľahlého píku pre protónové a jadrovo-jadrové zrážky. Pozadie v protónovo-protónových zrážkach je konštantné, v jadrovo-jadrových zrážkach je \-mo\-di\-fi\-ko\-va\-né kolektívnym tokom (flow).}
		\label{kor}
	\end{center}
\end{figure}

Po korekciách na účinnosť rekonštrukcie dráh asociovaných častíc, po normovaní na počet trigrovacích častíc a po odpočítaní pozadia, ktoré je v prípade protónovo-protónových zrážok konštantné a v prípade jadrovo-jadrových zrážok modifikované kolektívnym tokom, sa počítajú výťažky ("yields") pre jednotlivé intervaly priečnej hybnosti trigrovacích častíc. Výťažky, ktoré predstavujú priemerný počet asociovaných častíc na jednu trigrovaciu časticu pri danej priečnej hybnosti možno vypočítať ako:

\begin{equation}
Y_J^{\Delta\phi}=\int_{\Delta \phi_1}^{\Delta \phi_2} \frac{dN}{d\Delta \phi } d\Delta\phi 
\label{yield}
\end{equation} 

Touto metódou je možné počítať výťažky pre neidentifikované ako aj pre identifikované trigrovacie častice. Výber identifikovaných trigrovacích častíc je výhodný, pretože sa vďaka tomu dá študovať modifikácia fragmentačnej funkcie\footnote{Fragmentačná funkcia popisuje pravdepodobnosť, s akou vznikne daná častica s danou hybnosťou v jete.} v jadrovo-jadrových zrážkach oproti protónovým zrážkam. Vhodnými kandidátmi na trigrovacie častice sú podivné hadróny $\Lambda$ a $K^0_S$, pretože sú dobre identifikovateľné aj pri vysokých priečnych hybnostiach. 

\section{Súčasný prehľad problematiky}
\subsection{Experiment STAR na RHIC}
Na experimente STAR na urýchľovači RHIC boli študované dvojčasticové korelácie na priľahlej strane pre zrážky d+Au, Cu+Cu, Au+Au pri energii $\sqrt{s_{NN}}=$200 GeV zozbierané v rokoch 2003, 2004 a 2005 pre neutrálne podivné baryóny ($\Lambda , \bar{\Lambda}$) a mezóny ($K^0_S$) \cite{clanokstar}. Študované boli ako korelácie s identifikovanou trigrovacou časticou tak aj s identifikovanou asociovanou častiou. Skúmané boli závislosti výťažkov od centrality\footnote{Centralita predstavuje mieru prekryvu dvoch jadier pri zrážke. V literatúre zrážky s centralitou 0-5\% predstavujú maximálne prekrytie a ako periférne zrážky sú označované tie s centralitou 70-90\%.} zrážky a od priečnej hybnosti trigrovacej a asociovanej častice.

Neutrálne podivné (tzv.$V^0$ \footnote{Jedná sa o neutrálne slabo sa rozpadávajúce sa častice, ktorých topológia rozpadu pripomína písmeno V.}) častice boli identifikované na základe ich rozpadových kanálov:
\begin{equation}
\Lambda \rightarrow p + \pi^{-}  (BR=(63.9 \pm 0.5)\%)
\end{equation}
\begin{equation}
\bar{\Lambda} \rightarrow \bar{p} + \pi^{+}  (BR=(63.9 \pm 0.5)\%)
\end{equation}
\begin{equation}
K^0_S \rightarrow \pi^{+} + \pi^{-}  (BR=(68.95 \pm 0.14)\%)
\end{equation}

Rekonštruované $V^0$ sa museli nachádzať v intervale pseudorapidity ($-1$,1). Keďže neboli pozorované žiadne rozdiely medzi $\Lambda$ a $\bar{\Lambda}$, tieto korelácie boli skombinované, aby sa dosiahla väčšia štatistická významnosť.

Vyberané boli trigrovacie častice s vysokou $p_T$ a bolo počítané rozdelenie:
\begin{equation}
\frac{d^2N}{d\Delta \phi d\Delta \eta}(\Delta\phi,\Delta\eta) = \frac{1}{N_{trig}}\frac{d^2N_{raw}}{d\Delta \phi d\Delta \eta}\frac{1}{\varepsilon_{asoc}(\phi,\eta)}\frac{1}{\varepsilon_{pair}(\Delta\phi,\Delta\eta)},
\end{equation}
kde $\varepsilon_{asoc}(\phi,\eta)$ je korekcia na účinnosť rekonštrukcie jednotlivých častíc v TPC\footnote{Časovo-projekčná komora (Time Projection Chamber) - viac v \ref{textTPC}} a $\varepsilon_{pair}(\Delta\phi,\Delta\eta)$ je korekcia na párovú akceptanciu TPC. 

Účinnosť rekonštrukcie jednotlivých častíc bola určovaná ako funkcia $p_T$, $\eta$ a centrality simulovaním odozvy TPC na prechod častice. Účinnosť bola približne konštantná 75\% pre centrálne Au+Au zrážky, okolo 85\% pre periférne Cu+Cu zrážky a 89\% pre d+Au. Efektivita rekonštrukcie $\Lambda, \bar{\Lambda}$ a $K^0_S$ sa pohybovala od 8\% do 15\%, pričom narastala s hybnosťou a zmenšovala sa s veľkosťou systému. Systematická chyba spojená s účinnosťou rekonštrukcie bola odhadnutá na 5\%.

Na určenie párovej akceptancie bola použitá metóda mixing, keď sa trigrovacia častica vyberie z jedného prípadu a asociovaná z iného. Bolo požadované, aby mixovacie zrážky mali primárne vertexy od seba vzdialené maximálne 2cm a multiplicita prípadov sa líšila maximálne o 10 častíc. Jedna trigrovacia častica bola mixovaná s asociovanými časticami pochádzajúcimi z desiatich zrážok.

Výťažky boli počítané v intervaloch $|\Delta\phi|<0,78$ a $|\Delta\eta|<0,78$. Najprv bola urobená projekcia na os $\Delta\eta$:
\begin{equation}
\frac{dN}{d\Delta\eta}|_{\Delta\phi_1 \Delta\phi_2}= \int_{\Delta\phi_1}^{\Delta\phi_2}d\Delta\phi \frac{d^2N}{d\Delta \phi d\Delta \eta}
\end{equation}
Korelované\footnote{Korelované pozadie vzniká v dôsledku kolektívneho toku pri rozpínaní QGP.} aj nekorelované pozadie sa nemení ako funkcia $\Delta \eta$, preto bolo spoločne nafitované a následne odčítané ako konštantná funkcia. Systematická chyba merania spojená s určením pozadia bola 2\%. Výťažok bol počítaný ako:

\begin{equation}
Y^{\Delta\eta}_J=\int_{\Delta\eta_1}^{\Delta\eta_2}d\Delta\eta \frac{dN(\Delta\eta)}{d\Delta\eta}
\end{equation}
Na obrázku~\ref{clanokstar} je zobrazený výťažok ako funkcia priečnej hybnosti trigrovacej častice pre korelácie $K_S^0 - h$ a $\Lambda-h$ pre zrážky d+Au, Cu+Cu a Au+Au pri 200 GeV. Nie sú viditeľné žiadne zreteľné rozdiely vo výťažkoch medzi jednotlivými zrážkovými systémami. 

\begin{figure}[hbtp!]
	\centering
	\includegraphics[scale=0.5]{./Obrazky_praca/clanokstar.png}
	\caption{Závislosť výťažkov od priečnej hybnosti identifikovanej trigrovacej častice pre $1.5 GeV/c <p_T^{asoc}<p_T^{trig}$ v intervale $|\Delta\eta|<0.78$ ~\cite{clanokstar}}
	\label{clanokstar}
\end{figure}

\subsection{Experiment ALICE na LHC}
QGP, ktorá vzniká v zrážkach relativistických ťažkých iónov, je študovaná pomocou signatúr spojených s produkciou a vlastnosťami rôznych častíc. Jednou z týchto signatúr je modifikácia jetov vplyvom QGP. Jety s istými energiami dokonca nie sú v dátovej vzorke prítomné, pretože boli absorbované plazmou. Tento jav sa nazýva "jet quenching" (z angl. - zhášanie jetov) a bol prvýkrát pozorovaný pri štúdiu dvoj\-čas\-ti\-co\-vých korelácii pri zrážkach zlato-zlato pri energii $\sqrt{s_{NN}}=$130 GeV na urýchľovači RHIC~\cite{rhic}.

V kolaborácii ALICE bolo analyzovaných 14 miliónov olovených zrážok z roku 2010 pri energii $\sqrt{s_{NN}}=$ 2.76TeV a 37 miliónov protónových zrážok z marca 2011 pri rovnakej energii metódou dvojčasticových korelácií. Dáta boli analyzované pre intervaly $8<p^{trig}_{T}<15$ GeV/c a 3 GeV/c$<p^{assoc}_{T}<p_T^{trig}$ a merali sa pomery výťažkov v centrálnych zrážkach ku výťažkom v periférnych zrážkach ($I_{CP}$) pre rôzne centrality a pomery výťažkov v olovených ku protónovým zrážkam ($I_{AA}$) \cite{clanok}. Pomer $I_{AA}$ ukazuje vplyv média na konečné stavy hadrónov. 

Na obrázku~\ref{clanok2} sú zobrazené pomery $I_{AA}$ pre centrálne a periférne zrážky, pričom boli použité tri spôsoby odčítania pozadia. Potlačenie výťažku $I_{AA}\approx0.6$ pre protiľahlý pík centrálnych zrážok je považované za dôkaz straty energie v QGP. Viditeľné je aj zvýšenie pre priľahlý pík, $20-30\%$, ktoré nebolo pozorované pri nižších energiách zrážok na RHICu. Pre periférne zrážky nie sú zreteľné žiadne vplyvy plazmy, ako bolo očákavané. Podľa výsledkov tejto analýzy je pomer $I_{CP}$ v súlade s $I_{AA}$ pre centrálne zrážky (Obr.~\ref{clanok2}).



Počas tejto analýzy bolo prvý krát pozorované už skôr spomenuté navýšenie pomeru $I_{AA}$ pre centrálne zrážky, čo napovedá, že aj priľahlý partón bol ovplyvnený QGP. V publikácii~\cite{clanok} sú uvedené 3 faktory ovplyvňujúce pomer $I_{AA}$:
\begin{itemize}
	\item \textbf{zmena fragmentačnej funkcie}: vplyvom plazmy nesú hadróny v olovených zrážkach menšiu časť hybnosti pôvodného partónu ako v protónových zrážkach, čo vedie k zvýšenému počtu asociovaných častíc a $I_{AA}>1$
	\item \textbf{rôzna distribúcia počiatočných partónov}: zvýšený počet gluónov má rovnaký vplyv a vedie k potlačeniu počtu trigger častíc v danom $p_T$ intervale, z čoho plynie $I_{AA}>1$
	\item \textbf{posun $p_T$ spektra partónu}: pre danú $p_T$ trigrovacej častice, mal pôvodný partón vyššiu $p_T$ v olovených zrážkach ako v protónových, čo vedie k zvýšeniu $I_{AA}$
\end{itemize}

Predpokladá sa, že všetky tri efekty hrajú úlohu a prispievajú k nameranému výsledku.

\begin{figure}[hbtp!]
	\centering
	\includegraphics[scale=1]{./Obrazky_praca/clanok2.png}
	\caption{ Hore: $I_{AA}$ pre centrálne a periférne zrážky; dole: $I_{CP}$: Výsledky získané pomocou rôznych metód odčítania pozadia~\cite{clanok}}
	\label{clanok2}
\end{figure}



\chapter{Ciele práce} 
\begin{itemize}
	\item Vypracovanie všeobecného programu použiteľného na analýzu identifikovaných a neidentifikovaných dvojhadrónových korelácií na všetky typy zrážok pri rôznych energiách
	\item Testovanie kódu na dátach z protónovo-protónových zrážok pri 13 TeV (vysoká štatistika, nízke pozadie)
	\begin{itemize}
		\item Testovanie metódy na h-h koreláciach
		\item Analýza korelácií s identikovanými trigrovacími časticami (podivné častice)
		\item Porovnanie výsledkov s MC vygenerovanými dátami 
	\end{itemize}
\end{itemize}


\chapter{Metóda merania}

\section{Experiment ALICE}

ALICE (z anglického A Large Ion Collider Experiment – experiment na Veľkom iónovom zrážači) je jedným zo štyroch veľkých experimentov na LHC v CERNe. Zameriava sa najmä na výskum partónovej hmoty vznikajúcej v zrážkach ultrarelativistických ťažkých iónov. Táto hmota sa svojimi vlastnosťami podobá na teoreticky predpovedanú QGP, ktorou bol tvorený Vesmír tesne po Veľkom Tresku. V podmienkach LHC to zodpovedá približne 1 miliardtine sekundy veku vesmíru.



Detektor  ALICE umožňuje detekciu rôznych druhov častíc pri zrážkach protón - protón, ale predovšetkým umožňuje štúdium zrážok olovo - olovo pri extrémnych podmienkach (vysoká teplota, tlak a energia). Jeho rozmery dosahujú $16\times 16\times26 \rm{m^3}$ s váhou približne 10,000 ton~\cite{alice}. Celý detektor je umiestnený v solenoidálnom magnetickom poli s indukciou 0.5 T, ktoré spôsobuje zakrivenie dráh.

Experiment ALICE je tvorený viacerými detektormi, ako je zobrazené na obrázku~\ref{ALICE}. Každý z jednotlivých poddetektorov využíva inú technológiu a má svoju nezastúpiteľnú úlohu v celom detektorovom komplexe.
\newpage

\begin{figure}[hbtp!]
	\begin{center}
		\includegraphics[width=\textwidth]{./Obrazky_praca/ALICE.png}
		\caption{Experiment ALICE skladajúci sa z viacerých subdetektorov~\cite{aliceDetektor}}
		\label{ALICE}
	\end{center}
\end{figure}

\subsection{Vnútorný dráhový systém}

ITS (z anglického Inner Tracking System) je systém najvnútornejších detektorov, ktorý sa nachádza blízko miesta zrážok, a teda v oblasti s vysokou hustotou častíc (pre zrážky olovo-olovo až 50 dráh na $\mathrm{cm}^2$ ). Je tvorený šiestimi vrstvami kremíkových detektorov, ktoré slúžia najmä na detekciu primárneho vertexu\footnote{Miesto zrážky}, detekciu pile-up, určovanie sekundárneho vertexu slabých rozpadov a určovanie a rozlišovanie dráh jednotlivých častíc. 

Každá dvojica vrstiev využíva inú technológiu:
\begin{itemize}
	\item SPD (z angl. Silicon Pixel Detector - kremíkový pixlový detektor) tvoria dve najvnútornejšie vrstvy. Využíva reverzne zapojenú PN diódu s katódou rozdelenou na 2D mriežku. Každému štvorcu katódy (pixel) zodpovedá jeden “read -out”. Aktína časť je od čítacej oddelená “solid bump”. Vďaka jemnému deleniu katódy má tento typ vysoké rozlíšenie a podáva 2D informáciu o polohe prechodu častice. 
	\item SDD (z angl. Silicon Drift Detector - kremíkový driftový detektor) sú použité pre prostredné dve vrstvy. Tento typ polovodičového detektora využíva na určovanie súradníc prechodu častice meranie času odkedy častica preletela detektorom a ionizovala ho až kým sekundárne elektróny neprídu v špeciálne tvarovanom elektickom poli ku anóde. Druhá súpradnica je meraná pomocou rozloženia náboja na anóde.
	\item SSD (z angl. Silicon Strip Detector - kremíkový stripový detektor) sú aplikované na dvoch vonkajších vrstvách. V tomto prípade ide o reverzne zapojenú PN diódu, pričom katóda aj anóda sú rozdelené na tenké pásiky (stripy) v navzájom kolmom smere. Každý z týchto stripov má samostatný read-out, čo poskytuje dvojrozmernú informáciu o prechode častice. 
\end{itemize}
\subsection{Dopredné detektory}

Na oboch koncoch ITS barela sa nachádzajú tzv. dopredné detektory (forward detectors). Hlavnou úlohou vädšiny z nich je “triggering”. Každý z nich je zhotpovený z iného druhu detektora:
\begin{itemize}
	\item V0 je tvorený dvoma kruhovými časťami (V0A a V0C), pričom každá z nich sa náchadza na jednej stane ITS barela. Každý z kruhov je tvorený 32 scintilačnými počítačmi (4 sektory v radiálnom smere a 8 v azimutálnom) s posunom vlnovej dĺžky. 
	\item T0 meria on-line pozíciu primárneho vertexu, poskytuje hrubý odhad centrality pre zrážky olovo-olovo a určuje presný čas zrážky, čo je štartovací bod pre ToF detektor. Je tvorený dvanástimi cylindrickými Čerenkovovými počítačmi vyrobenými z taveného kremeňa usporiadanými v kruhu na oboch stranách ITS barela.
	\item FMD (z angl. Forward Multiplicity Detector - dopredný multiplicitný detektor), ako už vyplýva z názvu, úlohou tohto detektora je merať multiplicitné rozdelenie nabitých častíc a navyše je schpný zmerať aj multipliciné fluktuácie v jednotlivých zrážkach. Tento detektor je tvorený stripovými polovodičovými detektormi.	 
\end{itemize}

\subsection{Časovo-projekčná komora}
\label{textTPC}
Na pozorovanie dráh nabitých častíc produkovaných v zrážkach slúžia dráhové detektory. Hlavný z nich je TPC (z anglického Time Projection Chamber - časovo projekčná komora).

TPC je valec naplnený plynovou zmesou ( $Ne/CO_2/N_2$) s objemom $88m^3$ (obr.~\ref{tpc}) rozdelený centrálnou elektródou na dve driftovacie časti. Pozdĺž z-ovej osi je udržiavané rovnomerné elektrické pole o veľkosti 100kV. Nabité častice pri svojej ceste ionizujú plyn v TPC. Tým sú uvoľňované elektróny z molekúl plynu, ktoré po krátkom vychýlení následne driftujú k anódam na okraji valca. Na základe informácie o čase driftu elektrónov a o mieste ich dopadu sa rekonštruujú body v priestore, cez ktoré častica preletela. Signál je zosilnený lavínovým efektom v blízkosti anódy~\cite{alice1}, ktorá je tvorená mnohovláknovými proporcionálnymi komorami. Tie sú rozdelené na 18 sektorov v azimutálnom uhle a na 2 sektory v radiálnom smere. Homogénne magnetické pole v smere osi z, v ktorom je celý detektor umiestnený, umožňuje určenie hybnosti a elektrického náboja častíc. Z informácií o energetických stratách v TPC a v iných detektoroch je možné určiť aj druh častice. 

\begin{figure}[hbtp!]
	\begin{center}
		\includegraphics[width=\textwidth]{./Obrazky_praca/tpc.png}
		\caption{TPC naplnená 90\% neónom a 10\% oxidom uhličitým~\cite{TPCobr}}
		\label{tpc}
	\end{center}
\end{figure}

\subsection{TRD}
Na identifikáciu elektrónov slúži predovšetkým TRD (z angl. Transition Radiation Detector - detektor prechodového žiarenia). Skladá sa z radiátora a driftovacej komory, na ktorej konci sa nachádzajú mnohovláknové proporciálne komory (MWPC). Driftovacia komora je naplnená zmesou xenónu a oxidu uhličitého (85/15\%) a je v nej udržiavané homogénne elektrické pole s veľkosťou intenzity 700 V/cm. Veľkosť prechodového žiarenia závisí lineárne od relativistického faktora $\gamma$. Keďže elektróny sú reletivistické aj pre malé energie, vždy emitujú toto prechodové žiarenie. Preto je signál z elektrónov v MWPC zväčšený o signál z tohto prechodového žiarenia.

\subsection{ToF}
ToF detector (z angl. Time of Flight) je detektor zameraný na identifikáciu jednotlivých častíc. Má cylindrickú štruktúru a je rozdelený na 5 sektorov v smere osi z a na 18 sektorov v azimutálnom uhle. Pokrýva teda celý azimutálny uhol a inteval $45^\circ - 135^\circ$ v polárnom uhle. Dokopy je tvorený 1638 elementárnymi detektormi MRPC( z angl Multigap Resistive Plate Chamber - vysokoodporové platňové komory).

\subsection{Elektromagnetické kalorimetre}

Hlavným účelom eltromagnetických kalorimetrov je zaznamenať elektromagnetickú spŕšku iniciovanú fotónmi, elektrrónmi a pozitrónmi a na základe toho určovať energiu pôvodných častíc. Na detektore ALICE sa využívajú dva rôzne elektromagnetické kalorimetre:
\begin{itemize}
	\item EmCal (z angl. Electromagnetic Calorimeter - elektromagnetický kalorimeter ) je samplovací/vzorkovací kalorimeter, t.j. jeho základné tehličky sú tvorené striedajúcimi sa vrstvami olova a polystirénu. Olovo je tzv. pasívny materiál, v ktorom sa vyvýja spŕška a polystyrén je aktívny scintilačný materiál, z ktorého sa získava signál. Tvorí časť oblúka s akceptanciou pseudorapidy do 0.7 a $110^\circ$ v azimutálnom uhle.   
	PHOS (z angl. Photon Spectrometer - fotónový spektrometer) je homogénny kalorimeter, t.j. každá tehlička je vyrobená iba z jedného materiálu, scintiačného PWO kryštálu, ktorý je zároveň aktívnym aj pasívnym materiálom. Hlavným zameraním tohto kalorimetra je meranie spektra primárnych fotónov v iónových zrážkach, a tým určovanie teploty prvej fázy QGP. Detekčné kanály sú rozdelené do piatich modulov a pokrývajú $100^\circ$ v azimutálnom uhle a pseudorapidy v intervale -0.12 až 0.12. Operačná teplota je $-25^\circ C$.
\end{itemize}

\subsection{Hadrónové kalorimetre}
Hlavnou funkciou hadrónových kalorimetrov ZDC (z angl. Zero Degree Calorimeter),ktoré sú umiestnené 115 cm od IP na každej strane v blízkosti zväzkovej trubice, je meranie centrality olovených zrážok pomocou spektátorov\footnote{Sú to protóny a neutróny, ktoré sa na zrážke nezúčastnili. Čím väčší počet takýchto častíc je nameraných v ZDC, tým by sa molo jednať o periférnejšiu zrážku.}. Na každej strane sa nachádza jeden protónový a jeden neutrónový kalorimeter, pričom protónový je trocha odklonený od zväzkovej trubice, protože protóny ako nabité častice sú ovplyvnené magnetickým poľom. V oboch prípadoch sa jedná o vzorkovacie kalorimetre. Aktívnym matriálom sú oplické káble vkladané do wolfrámových (nautrónové kalorimetre) a mosadzných (protónové kalorimetre) platní, ktoré majú funkciu pasívneho materiálu. 

\subsection{Miónový spektrometer}
Súčasťou detektora ALICE je aj miónové rameno. Princíp fungovania je na obrázku~\ref{mion}. 90 cm od interakčného bodu sa nacháza hdrónový absorbér, krorý znižuje tok hadrónov o faktor 100. Tento absorbér bol vyrobený zo špeciálnych materiálov, aby spektrum invariantnej hmotnosti miónov bolo ovplyvnené čo najmenej. Za absorbérom sa nachádza 5 dráhových detektorov, pričom stredný z nich sa nechádza v elektrickom poli generovanom dipólovým magnetom. Toto magnetické pole slúži na zakrivenie dráh miónov, čo umožňuje určovanie ich hybnosti. Ďalšou časťou miónového spektrometra je 120 cm hrubá železná stena, cez ktorú neprejdú iné častice ako mióny. Za ňou sa nachádzajú trigrovacie komory. Ak je v nich generovaný signál, znamená to, že sa jedná o mións vlastnosťami, ktoré boli zmerané v predošlých dráhových detektoroch.

\begin{figure}[hbtp!]
	\begin{center}
		\includegraphics[width=\textwidth]{./Obrazky_praca/mion.png}
		\caption{Schéma miónového ramena na detektore ALICE}
		\label{mion}
	\end{center}
\end{figure}

\section{ROOT}
ROOT je vývojové prostredie napísané v objektovo orientovanom programovacom jazyku C++ určené na analýzu dát. Užívateľovi poskytuje veľký počet funkcionalít spojených so spracovávaním veľkého množstva dát, napr. histogramy, funkcie pre štatistickú analýzu, vizualizáciu zrážok a bohatý matematický aparát. Užívateľ má taktiež možnosť vytvorenia vlastných tried a knižníc vhodných pre svoju analýzu. ROOT je možné prepojiť s už existujúcimi programovacími jazykmi a štatistickými programami, ako sú Python, Ruby, R a Mathematica~\cite{root}.

\section{AliRoot}
AliRoot je oficiálnym analyzačným softvérom kolaborácie ALICE, základom ktorého je ROOT rozšírený o špeciálne triedy, ktoré sa používajú na rekonštrukciu, simuláciu a analýzu dát v experimente ALICE. Jeho štruktúra, triedy a funkcie sú vytvárané v súlade s~potrebami analýzy dát nameraných na experimente ALICE. Z toho vyplýva aj delenie jeho tried na všeobecné a triedy využívané pre rekonštrukcii, simulácii a analýze. 

Pomocou AliRootu je možné spracovávať dáta v rôznych formátoch:
\begin{itemize}
	\item \textbf {raw dáta} - signály z jednotlivých detektorov priradené k jednotlivým zrážkam, ktoré sa ďalej spracovávajú na gride\footnote{Grid je celosvetová počítačová sieť, na ktorej sa rekonštruujú a analyzujú dáta. Grid experimentu ALICE sa nazýva AliEn a obsahuje okolo 100000 CPU jadier.} 
	\item \textbf {ESD (Event Summary Data) dáta} - spracované ale nefiltrované raw dáta, obsahujú objekty, ktoré sa vypočítali pomocou raw dát, ako napríklad druhy nabitých častíc alebo slabé neutrálne rozpady. K jednotlivým objektom sú priradené ich vlastnosti ako hmotnosť, hybnosť, energia, elektrický náboj a pod..
	\item \textbf {AOD (Analysis Object Data) dáta} - fitrované ESD dáta vhodné skoro pre všetky druhy analýz, tvoria približne 10\% pôvodnej veľkosti ESD dát, použité aj v tejto analýze
\end{itemize}

Pre všetky tieto druhy dát sú v AliRoote vytvorené špeciálne triedy s podobnými funkciami. Okrem dát zozbieraných z detektorov, AliRoot umožňuje spracovanie aj simulovaných dát, pre ktoré tiež existujú samostatné triedy. Na simulácie sa využívajú rôzne Monte Carlo (MC) generátory zrážok:
\begin{itemize}
	\item HIJING - generovanie jadrovo-jadrových zrážok, založený na QCD a Lundskom modeli fragmentácie jetov
	\item PYTHIA, Herwig, Phojet - generovanie protónovo-protónových zrážok
\end{itemize}

Rekonštrukcia simulovaných dát (raw, ESD, AOD) sa uskutočňuje pomocou rovnakého kódu ako rekonštrukcia reálnych dát.

\chapter{Výsledky}

\section{Spracované dáta a selekčné kritéria}
V analýze boli spracovávané dáta z protónovo-protónových zrážok pri energii $\sqrt{s_{NN}}=13$~TeV z roku 2015 s označením LHC15f. Analyzované dáta boli vo formáte AOD. Dokopy bolo analyzovaných $1.17\times10^7$ zrážok a celý výpočet prebehol na gride.
MC produkcia bola zviazaná s týmito daátami a boli použité MC produkcie z 3 rôznych generátorov - PYTHIA6 (LHC15g3c3), PYTHIA8 (LHC15g3a3), EPOS (LHC16d3). 

Analyzované boli iba tie zrážky, ktorých primárny vertex\footnote{miesto zrážky dvoch protónov} bol vzdialený maximálne 10 cm od geometrického stredu detektora. Nabité primárne hadróny museli spĺňať selekčné kritériá optimalizované pre tento druh analýzy.

Na obrázku~\ref{schema} je zobrazená schéma rozpadu V0 častice, konkrétne $\Lambda$. Pri selekcii jednotlivých V0 častíc sme sa zamerali na parametre:
\begin{itemize}
	\item DCA (Distance of Closest Approach) V0 Neg. daughter - najkratšia vzdialenosť v priestore negatívne nabitej dcéry k primárnemu vertexu
	\item DCA V0 pos. - najkratšia vzdialenosť pozitívne nabitej dcéry
	\item DCA between daughters - najkratšia vzdialenosť v priestore medzi dcérami V0 častice
	\item V0 2D Decay Radius  - je vzdialenosť bodu rozpadu V0 (sekundárny vertex) a primárneho vertexu zrážky v $xy$ rovine
	\item Pointing angle $\theta_{PA}$ - uhol medzi úsečkou spájajúcou primárny vertex zrážky a bod rozpadu V0 a vektorom celkovej hybnosti V0 častice
	\item lifetime - doba života
	\item rapidita - funkcia celkovej energie a pozdĺžnej zložky hybnosti častice
\end{itemize} 
Všetky hodnoty selekčných kritérii sú zosumarizované v tabuľke~\ref{tabulka}.

\begin{figure}[hbtp!]
	\centering
	\includegraphics[scale=0.9]{./Obrazky_praca/SchemaV0.png}
	\caption{Schéma topologickej selekcie pre V0 častice~\cite{schema}}
	\label{schema} 
\end{figure}

\begin{table}[hbtp!]
	\begin{center}
		\begin{tabular}{|c|c|}
			\hline
			 \multicolumn{2}{|c|}{Selekčné kritéria pre V0}  \\ \hline
			$K^0_S$ & $\Lambda$ + $\bar{\Lambda}$ \\ \hline
			$cos\theta_{PA} >0.97$ & $cos\theta_{PA} >0.995$  \\ \hline
			$|y|<0.5$ & $|y|<0.5$   \\ \hline
			DCA Neg$>$0.06 cm & DCA Neg$>$0.06 cm \\ \hline
			DCA Pos$>$0.06 cm & DCA Pos$>$0.06 cm \\ \hline
			DCA V0 daughters$<$ 1 cm & DCA V0 daughters $<$1 cm \\ \hline
			lifetime$<$20 cm & lifetime$<$30 cm \\ \hline
			V0 2D Decay Radius $>$ 0.5 & V0 2D Decay Radius $>$ 0.5\\ 
			\hline
		\end{tabular}
		\caption{Tabuľka selekčných kritérii pre V0 častice}
		\label{tabulka}
	\end{center}
\end{table}
\newpage
Selekčné kritéria boli aplikované aj na dcérske častice pochádzajúce z $V^0$. Rekonštruované dráhy nabitých dcérskych častíc $V^0$ prešli najmenej cez 70 rovín ("padrows") v TPC. Pomer počtu reálne prejdených a teoreticky možných rovín v TPC musel byť väčší ako 0.8. Posledným kritériom výberu dcérskych častíc bola ich identifikácia pomocou straty energie v TPC pre V0 častice s priečnou hybnosťou menšou ako 1 GeV, kde strata energie kandidáta na dcérsku časticu V0 musela byť v intervale do $3\sigma$ od teoreticky vypočítanej straty energie pre danú časticu.

Posledným kritériom potrebným na štúdium dvojčasticových korelácií je priečna hybnosť. Trigrovacia častica musela mať priečnu hybnosť väčšiu ako 4 GeV/c a asociovaná sa musela nachádzať v intervale 2 GeV/c$<p_T^{asoc}<p_T^{trig}$.

\section{Korekcie}
Okrem jednotlivých selekčných kritérii bolo potrebné pri analýze použiť aj tri typy korekcií. 

Prvou je korekcia na konečnú akceptanciu detektora. Túto korekciu sme urobili pomocou metódy mixing. Trigrovacia častica bola z jedného prípadu a asociovaná častica z iného nezávislého prípadu. Kvôli väčšej štatistike sme jednu trigrovaciu časticu mixovali s minimálne piatimi prípadmi, pričom oba prípady museli mať z-ovú zložku primárneho vertexu rovnakú v rámci 1 cm a spadať do rovnakého multiplicitného binu. Mixovaním získané rozdelenie tvaru kombinatorického pozadia je na obrázku~\ref{pozadie}. Toto rozdelenie bolo následne normované na hodnotu najvyššieho binu rozdalenia. Za predpokladu, že účinnosť rekonštrukcie páru v najvyššom bine je 100\%, získame tak rozdelenie účinnosti rekonštrukcie párov v priestore $(\Delta \phi,\Delta\eta)$.

Druhou je korekcia na účinnosť rekonštrukcie asociovaných častíc v detektore. Pri tejto korekcii boli využité dáta nasimulované metódou Monte Carlo, ktoré zodpovedajú energii aj druhu zrážok spracovávaných dát. Generované dáta prešli \-re\-kon\-štruk\-ciou, pri ktorej boli použité rovnaké selekčné kritéria ako pre reálne dáta. Účinnosť re\-kon\-štruk\-cie sa definuje ako podiel počtu zrekonštruovaných častíc ku počtu generovaných v závislosti od priečnej hybnosti a  pseudorapidity asociovanej častice a z-ovej súradnice primárneho vertexu, pričom každá rekonštruovaná častica musí byť spojená s generovanou časticou.  

Poslednou je korekcia na účinnosť rekonštrukcie trigrovacích častíc. Bola robená podobne ako pre asociované častice v závislosti od priečnej hybnosti a  pseudorapidity trigrovacej častice a z-ovej súradnice primárneho vertexu. Na obrázku \ref{uc} je zobrazená účinnosť rekonštrukcie pre jednotlivé typy častíc v závislosti od ich priečnej hybnosti.

\begin{figure}[hbtp!]
	\centering
	\includegraphics[scale=0.5]{./Obrazky_praca/pozadie.png}
	\caption{Príklad tvaru kombinatorického pozadia, t.j. častice v páre spolu fyzikálne nesúvisia, pri dvojčasticových koreláciách namodelovaného metódou mixing}
	\label{pozadie}
\end{figure}

\begin{figure}[hbtp!]
	\centering
	\includegraphics[scale=0.5]{../histogramy/vytazky/ucinnost.png}
	\caption{Účinnosť rekonštrukcie jednotlivých častíc v yávislosti od ich priečnej hybnosti}
	\label{uc}
\end{figure}

\section{MC closure test}

Monte Carlo closure test je v princípe skúška správnosti. Použitá tu bola kombinácia dvoch produkcií MC (LHC15g3a3 a LHC15g3c3) kvôli zvýšeniu štastistiky. V rámci tohto testu sa pozeráme na korelačné funkcie hadrónov, ktoré boli vygenerované a na korelačné funkcie MC zrekonštruovaných hadrónov, na ktoré boli aplikované už skôr spomínané korekcie. Pre generované ako aj rekonštruované dáta sme urobili korelačné funkcie v závislosti od priečnej hybnosti  a od druhu trigrovacej častice. Následne sme tieto dvojrozmerné funkcie generovaných častíc predelili funkciami rekonštruovaných častíc. Z týchto dvojrozmerných pomerov sme potom robili projekcie na jednotlivé osi, pričom prázdne biny boli ignorované, a tie sme fitovali konštantnou funkciou. Hodnoty jednotlivých fitov sme zobrazili v závislosti od priečnej hybnosti a druhu trigrovacej častice Obr \ref{fitMC} . Očakávaný výsledok bol v blízkom okolí 1, ktorý sme dosiahli pre korelácie, kde trgrovacími časticami boli nabité hadróny. Tento výsledok znamená, že s aplikovanými korekciami sme dokázali zrekonštruovať situáciu, len minimálne ovplyvnenú detektorom.
V prípade V0 ako trigrovacích častíc je fit viac posunutý od 1. Ako je vidieť na jednotlivých projekciách (Obr. \ref{K0phi}  až Obr.\ref{heta}  ), projekcia na $\Delta \phi$ je približne konštantná a rovnako posunutá od hodnoty 1 pre $K^0_S$ ako aj pre $\Lambda$. Pri projekcii na $\Delta \eta$ vidíme rovnaké správanie v okolí 0, teda v okolí korelačného píku, ale na okrajoch hodota pomeru klesá. Tento efekt by mohol byť spôsobený detektorom a bude zarátaný v rámci systematickej chyby. 

\begin{figure}
	\centering
	\begin{subfigure}{0.5\textwidth}
		\centering
		\includegraphics[width=1.1\linewidth]{../histogramy/PorovnanieMC/Projekcie/fitPhi.png}
		\caption{}
		\label{fitPhi}
	\end{subfigure}%
	\begin{subfigure}{0.5\textwidth}
		\centering
		\includegraphics[width=1.1\linewidth]{../histogramy/PorovnanieMC/Projekcie/fitEta.png}
		\caption{}
		\label{fitEta}
	\end{subfigure}
	\caption{Hodnota fitu pre jednotlivé $p_T$ biny.}
	\label{fitMC}
\end{figure}

\begin{figure}
	\centering
	\includegraphics[width=1\linewidth]{../histogramy/PorovnanieMC/Projekcie/15a+c_K0_phi.png}
	\caption{Projekcia na os $\Delta \phi$ pre korelácie $K^0_S - h$}
	\label{K0phi}
\end{figure}

\begin{figure}
	\centering
	\includegraphics[width=1\linewidth]{../histogramy/PorovnanieMC/Projekcie/15a+c_K0_eta.png}
	\caption{Projekcia na os $\Delta \eta$ pre korelácie $K^0_S - h$}
	\label{K0eta}
\end{figure}
\begin{figure}
\centering
\includegraphics[width=1\linewidth]{../histogramy/PorovnanieMC/Projekcie/15a+c_Lam_phi.png}
\caption{Projekcia na os $\Delta \phi$ pre korelácie $(\Lambda + \bar{\Lambda})- h$}
\label{Lamphi}
\end{figure}
\begin{figure}
	\centering
	\includegraphics[width=1\linewidth]{../histogramy/PorovnanieMC/Projekcie/15a+c_Lam_eta.png}
	\caption{Projekcia na os $\Delta \eta$ pre korelácie $(\Lambda + \bar{\Lambda})- h$}
	\label{Lameta}
\end{figure}

\begin{figure}
	\centering
	\includegraphics[width=1\linewidth]{../histogramy/PorovnanieMC/Projekcie/15a+c_hh_phi.png}
	\caption{Projekcia na os $\Delta \phi$ pre korelácie $h - h$}
	\label{hphi}
\end{figure}

\begin{figure}
	\centering
	\includegraphics[width=1\linewidth]{../histogramy/PorovnanieMC/Projekcie/15a+c_hh_eta.png}
	\caption{Projekcia na os $\Delta \eta$ pre korelácie $h - h$}
	\label{heta}
\end{figure}



\section{Výťažky}

Na získanie výťažkov sme najprv z dvojrozmerného rozdelenia $\frac{d^2N}{d\Delta \phi d\Delta \eta}$, ktoré bolo podelené rozdelením získaným z metódy mixing a korigované na efektivitu rekonštrukcie asociovaných aj trigrovacích častíc, projektovali jednorozmerné histogra\-mi pozdĺž osi $\Delta\phi$ pre rôzne intervaly priečnej hybnosti trigrovacej častice pre jednotlivé druhy trigrovacích častíc. Keďže medzi  výsledkami pre trigrovacie častice $\Lambda$ a $\bar{\Lambda}$ neboli viditeľné žiadne signifikantné rozdiely, ich rozdelenia boli spojené kvôli zmenšeniu štatistických chýb. Jednorozmerné rozdelenia sme následne normovali na počet trigrovacích častíc, ktorý bol taktiež korigovaný na efektivitu rekonštrukcie. 

Pozadie sme určovali pomocou konštantnej funkcie. Jej hodnota bola vypočítana ako aritmetický priemer hodnoty rozdelenia v troch binoch okolo 1 a troch binoch okolo $-1$. Výsledné rozdelenia sú zobrazené na obrázkoch \ref{K0} až \ref{hh}. 

\begin{figure}[hbtp!]
	\centering
	\includegraphics[width=\textwidth]{./Obrazky_praca/DeltaPhiKH.png}
	\caption{Rozdelenie $\Delta \phi$ pre trigrovaciu časticu $K^0_S$ po odčítaní pozadia pre rôzne intervaly $p_T^{trig}$}
	\label{K0}
\end{figure}

\begin{figure}[hbtp!]
	\centering
	\includegraphics[width=\textwidth]{./Obrazky_praca/DeltaPhiLH.png}
	\caption{Rozdelenie $\Delta \phi$ pre pre trigrovacie častice $\Lambda$ a $\bar{\Lambda}$  po odčítaní pozadia pre rôzne intervaly $p_T^{trig}$}
	\label{lambda}
\end{figure}

\begin{figure}[hbtp!]
	\centering
	\includegraphics[width=\textwidth]{./Obrazky_praca/DeltaPhiHH.png}
	\caption{Rozdelenie $\Delta \phi$ pre nabitý hadrón ako trigrovaciu časticu po odčítaní pozadia pre rôzne intervaly $p_T^{trig}$}
	\label{hh}
\end{figure}
\newpage

Výťažky sme počítali pomocou vzťahu~\ref{yield} v intevale $-0.9$ až 0.9 pre priľahlý pík a v intervale $\pi\pm1.5$ pre protiľahlý pík. Závislosti výťažkov pre priľahlý a protľahlý pík od priečnej hybnosti trigrovacej častice sú zobrazené na obrázkoch \ref{near} a \ref{away}.

\begin{figure}[hbtp!]
	\centering
	\includegraphics[width=\textwidth]{./Obrazky_praca/vytazok_near.png}
	\caption{Závislosť výťažkov od priečnej hybnosti trigrovacej častice pre priľahlý pík}
	\label{near}
\end{figure}

\begin{figure}[hbtp!]
	\centering
	\includegraphics[width=\textwidth]{./Obrazky_praca/vytazok_away.png}
	\caption{Závislosť výťažkov od priečnej hybnosti trigrovacej častice pre protiľahlý pík}
	\label{away}
\end{figure}

\chapter{Diskusia}
 
Na grafoch závislosti výťažkov od priečnej hybnosti trigrovacej častice je viditieľné, že výťažok rastie so zväčšujúcou sa $p_T$ trigrovacej častice pre priľahlý aj pre protiľahlý pík a pre všetky druhy trigrovacích častíc. Tento výsledok bol očákavaný, pretože trigrovacia častica s vysokou $p_T$ pochádza z jetu s vysokou energiou, ktoré majú vyššiu multiplicitu ako jety s nižšou energiou, a teda sa tam nachádza väčší počet asociovaných častíc. 

Výťažky protiľahlých píkov sú nižšie ako priľahlých. Tento efekt môže byť spôsobený tým, že protiľahlý jet nie je viditeľný v detektore v prípade, že sa nachádzá v intervale pseudorapidity mimo akceptancie detektora. 

Čo sa týka veľkosti výťažkov pre jednotlivé trigrovacie častice, hoci nevieme stanoviť jednoznačný trend kvôli veľkým štatistickým chybám a fluktuáciám výťažkov, je viditeľné výrazné zvýšenie výťažkov pre korelácie s trigrovacími časticami $\Lambda$ a $\bar{\Lambda}$ v priľahlom aj protiľahlom píku pre najvyššie $p_T$ trigrovacích častíc. Toto zvýšenie by mohlo byť spôsobené tým, že baryóny s väčšou pravdepodobnosťou vznikajú v gluónových jetoch, ktoré majú vyššiu multiplicitu, a teda aj vyšší výťažok. Druhým vysvetlením je fluktuácia spôsobená systematickými chybami, ktoré zatiaľ neboli do výsledkov započítané. Toto vysvetlenie podporuje aj porovnanie výťažkov pre priľahlý pík s MC dátami(obr. \ref{porovnanieL}), kde je viditeľný odklon výťazku od MC hodnoty až 35\% (za predpokladu, že MC dobre popisuje počty podivých častíc v jetoch).    

Výťažky pre priľahlý aj protiľahlý pík pre všetky trigrovacie častice sme porovnali s MC dátami pre rôzne typy generátorov, ktoré sú zobrazené na obrázkoch~\ref{porovnanie} až \ref{porovnanieh}. Najlepšie dáta popisuje generátor PYTHI6 - Perugia, ako je možné vidieť na spodných častiach obrázkov, kde sú zobrazené pomery výťažkov MC ku výťažkom z dát.
%Pomer výťažkov pre generované \-re\-konšt\-ru\-o\-va\-né dáta a namerarané dáta je približne rovný 1 pre všetky trigrovacie častice a pre všetky intervaly priečnej hybnosti. To ukazuje, že PYTHIA veľmi dobre popisuje fragmentáciu nabitých častíc ako aj podivných častíc. 

\begin{figure}[hbtp!]
	\centering
	\begin{subfigure}{0.5\textwidth}
		\centering
		\includegraphics[width=1.1\linewidth]{../histogramy/vytazky/K0_prilahly.png}
		\caption{}
	\end{subfigure}%
	\begin{subfigure}{0.5\textwidth}
		\centering
		\includegraphics[width=1.1\linewidth]{../histogramy/vytazky/K0_protilahly.png}
		\caption{}
	\end{subfigure}
	\caption{Porovnanie závislosti výťažkov od priečnej hybnosti trigrovacej častice pre priľahlý pík (a) a pre protiľahlý pík (b) s trigrovacou časticou $K^0_S$ reálnych dát s rekonštruovanými MC dátami pochádzajúcich z rôznych generátorov}
	\label{porovnanie}
\end{figure}

\begin{figure}[hbtp!]
	\centering
\begin{subfigure}{0.5\textwidth}
	\centering
	\includegraphics[width=1.1\linewidth]{../histogramy/vytazky/lam_prilahly.png}
	\caption{}
\end{subfigure}%
\begin{subfigure}{0.5\textwidth}
	\centering
	\includegraphics[width=1.1\linewidth]{../histogramy/vytazky/lam_protilahly.png}
	\caption{}
\end{subfigure}
	\caption{Porovnanie závislosti výťažkov od priečnej hybnosti trigrovacej častice pre priľahlý pík (a) a pre protiľahlý pík (b) s trigrovacími časticami $\Lambda$ a $\bar{\Lambda}$ reálnych dát s rekonštruovanými MC dátami pochádzajúcich z rôznych generátorov}
	\label{porovnanieL}
\end{figure}

\begin{figure}[hbtp!]
	\centering
	\begin{subfigure}{0.5\textwidth}
		\centering
		\includegraphics[width=1.1\linewidth]{../histogramy/vytazky/hh_prilahly.png}
		\caption{}
	\end{subfigure}%
	\begin{subfigure}{0.5\textwidth}
		\centering
		\includegraphics[width=1.1\linewidth]{../histogramy/vytazky/hh_protilahly.png}
		\caption{}
	\end{subfigure}
	\caption{Porovnanie závislosti výťažkov od priečnej hybnosti trigrovacej častice pre priľahlý pík (a) a pre protiľahlý pík (b) s~nabitým hadrónom ako trigrovacou časticou reálnych dát s~rekonštruovanými MC dátami pochádzajúcich z rôznych generátorov}
	\label{porovnanieh}
\end{figure}


\chapter{Záver}

V tejto práci sme vytvorili uviverzálny kód, ktorý je možné použiť na analýzu rôznych typov dát pomocou metódy dvojčasticových korelácií. Ten sme otestovali na dvojhadrónových koreláciách s neidentifikovanou trigrovacou časticou v protónovo-protónových dátach nameraných pri energii 13 TeV na experimente ALICE na urýchľovači LHC a aplikovali na korelácie s identifikovanými podivnými trigrovacími časticami na rovnakých dátach.

V ďalšej práci by sme chceli študovať rôzne zdroje systematických chýb a neskôr tento kód použiť na oloveno-olovené zrážky, kde sa vyskytuje QGP a merať tak jej vplyv na jednotlivé výťažky. 



%
%\bibliography{dp} %berie sa z dp.bib

\renewcommand{\bibname}{Zoznam pou�itej literat�ry}
\begin{thebibliography}{}
\bibitem{1}
COUGHLAN, G, J DODD a Ben GRIPAIOS. \textit{The ideas of particle physics: an introduction for scientists}. 3rd ed. /. Cambridge: Cambridge University Press, c2006, 254 p. ISBN 978-052-1677-752.
\bibitem{tetra}
AAIJ, R., B. ADEVA, M. ADINOLFI, et al. Observation of the Resonant Character of the Z(4430)$^{-}$ State \textit{Physical Review Letters}. 2014, \textbf{112}(22). ISSN 0031-9007. https://link.aps.org/doi/10.1103/PhysRevLett.112.222002
\bibitem{2}
AAIJ, R., B. ADEVA, M. ADINOLFI, et al. Observation of $J / \psi$ p Resonances Consistent with Pentaquark States in $\Lambda_b^0 \rightarrow J / \psi K^{-}p$ Decays. Physical Review Letters [online]. 2015, 115(7),  ISSN 0031-9007. https://link.aps.org/doi/10.1103/PhysRevLett.115.072001
\bibitem{3}
BETHKE, S. Experimental tests of asymptotic freedom. \textit{Progress in Particle and Nuclear Physics}. 2007, extbf{58}(2): 351-386. DOI: 10.1016/j.ppnp.2006.06.001. ISSN 01466410.  http://linkinghub.elsevier.com/retrieve/pii/S0146641006000615
\bibitem{4}
HE, Yuncun, Ivan VITEV a Ben-Wei ZHANG. 
\textit{ $O(\alpha)$ Analysis of Inclusive Jet and Di-jet Production in Heavy Ion Reactions at the Large Hadron Collider}. Physics Letters B [online]. 2012, 713(3), ISSN 03702693. http://linkinghub.elsevier.com/retrieve/pii/S037026931200603X
\bibitem{5}
ANDERSSON, Bo. \textit{The Lund model}. New York: Cambridge University Press, 2005, 4671p. ISBN 0521017343.
\bibitem{clanokstar}
ABELEV, B., L. ADAMCZYK, J. K. ADKINS, et al. Near-side azimuthal and pseudorapidity correlations using neutral strange baryons and mesons in d+Au, Cu+Cu and Au+Au collisions at $\sqrt{s_{NN}}=200$ GeV \textit{Physical Review C}. 2016, \textbf{94}(1). ISSN 2469-9985. https://link.aps.org/doi/10.1103/PhysRevC.94.014910
\bibitem{6}
http://physics.stackexchange.com/questions/155327/zero-net-force-on-grass-seeds-is-this-a-uniform-field
\bibitem{7}
http://hepg.sdu.edu.cn/THPPC/reports/seminar2009/0316\_lund\_string\_model.pdf
\bibitem{8}
https://en.wikipedia.org/wiki/Color\_confinement
\bibitem{clanok}
AAMODT, K., B. ABELEV, A. ABRAHANTES QUINTANA, et al. Particle-Yield Modification in Jetlike Azimuthal Dihadron Correlations in Pb-Pb Collisions at $\sqrt{s_{NN}} = 2.76$ TeV \textit{Physical Review Letters}. 2012, \textbf{108}(9). ISSN 0031-9007. https://link.aps.org/doi/10.1103/PhysRevLett.108.092301
\bibitem{root}
https://root.cern.ch/about-root
\bibitem{alice}
http://home.cern/about/experiments/alice

\bibitem{alice1}
http://aliceinfo.cern.ch/Public/en/Chapter2/Chap2\_TPC.html

\bibitem{rhic}
ADLER, C., Z. AHAMMED, C. ALLGOWER, et al. Centrality Dependence of High-$p_T$ Hadron Suppression in Au+Au Collisions at $\sqrt{s_{NN}} = 130$ GeV. \textit{Physical Review Letters}. 2002, \textbf{89}(20), ISSN 0031-9007, https://link.aps.org/doi/10.1103/PhysRevLett.89.202301
\bibitem{schema}
https://cds.cern.ch/record/2030272
\bibitem{aliceDetektor}
http://aliceinfo.cern.ch/ArtSubmission/node/716
\bibitem{TPCobr}
http://aliceinfo.cern.ch/Public/en/Chapter2/Chap2\_TPC.html
\bibitem{glasma}
GELIS, F., Color glass condensate and Glasma \textit{International Journal of Modern Physics A}. 2012, https://arxiv.org/pdf/1211.3327.pdf
\bibitem{phasDiagram}
MARTINEZ GARCIA, G. Advances in Quark Gluon Plasma, https://arxiv.org/pdf/1304.1452.pdf
\bibitem{vyvoj}
https://indico.cern.ch/event/634050/attachments/1477336/2332417/HeavyIons\_Part1.pdf
\bibitem{zvysenie}
LIETAVA R.  et al. (WA97 Collaboration) Strangeness enhancement at mid-rapidity in Pb–Pb collisions at 158 A GeV/c. \textit{J. Phys. G}. 1999,  \textbf{25}(181). http://cds.cern.ch/record/380839/files/cer-000306426.pdf 
\end{thebibliography}
%
\end{document}